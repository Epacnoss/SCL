\documentclass[11pt, oneside]{article}   	% use "amsart" instead of "article" for AMSLaTeX format
\usepackage{graphicx}				% Use pdf, png, jpg, or eps§ with pdflatex; use eps in DVI mode
\usepackage[top=2cm, left=2cm, right=1.5cm]{geometry}

\title{Development Indicators}
\author{Jack Maguire}

\begin{document}
\maketitle



\section{Glossary Definitions}
\subsection{GDP}
Gross Domestic Product (GDP) is the value of one country's production in one year. This takes into account everything produced, from fruits to computers. Where it is sold is not of matter, but only that it is made in that country. It also takes into account services, like providing medical or legal advice.

\subsection{GNI}
Gross National Income (GNI) is the value of all the goods and services (AKA the GDP) plus all the income from UK residents working abroad, but minus all the income of all foreign residents working in the UK.

\subsection{HDI}
Human Development Index (HDI) is a combined statistic of other statistics like infant mortality, GDP per capita, and literacy rates in order to give a balanced statistic that shows lots about a country.

\subsection{Indicators of Development}
These are statistics like People per Docter or GDP that can be used to tell how developed a country is.

\subsection{Infant Mortality Rates}
This is how many infants have died per 1000, and can be used to measure a country's health service, although to get a full measure, other statistics are needed like Life expectancy.

\subsection{Life Expectancy}
The Life Expectancy is how long an average person in the country will live for. Normally separate figures are given for men and women.

\subsection{Social Development}
Development relating to social issues like Death Rates, or Literacy.


\section{Questions}
\subsection{What is meant by social development?}
Social Development is Development relating to social things, rather than economic. Things like Literacy Rates, or Infant Mortality.


\subsection{How is development measured?}
Development is measured using statistics like GNI, GDP Per Capita, or HDI. These figures come from things like tax reports or censuses. There are also more wide terms like LIC or HIC which measure economic development.



\end{document}  