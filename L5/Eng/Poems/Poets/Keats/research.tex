\documentclass[11pt, oneside]{article}
\usepackage[top=1cm, left=2cm, right=1.5cm]{geometry}
\usepackage[backend=biber, style=verbose-ibid]{biblatex}
\addbibresource{$BIB}
\title{John Keats Research}

\author{Jack Maguire}

\begin{document}
\maketitle

\section {Where did John Keat Live}
All according to:\footnote{\textcite{wiki-keats}}
He was born on the 31st of October 1795, in Moorgate London.
After school, he thought he wanted to become a doctor, so he went to train at Guys Hospital.
Eventually, he moved into Wentworth Palace, after deciding he was to be a poet.
He eventually caught tuberculosis, and lived in Rome for his last months, dying on the 23rd of February 1821.

\section {Who were his poet friends}
Again, all according to:\footnote{\textcite{wiki-keats}}
Keats didn't get much of a chance to meet lots of other poets, but he met a few, starting by having his poem published by Leigh Hunt in May 1816.


\section{How do the ideas of Romanticism fit in with this poem}
There are lots of ways Keats fits ideas of Romanticism into his poem. Firstly, is his selective mentioning of only nature, rather than anything to do with man, such as choosing to give more attention to getting stoned on Poppies rather than the granary or the farmers house.
He also mentions that, the furrow has been 'half-reap'd', but rather than chastising the farmer, he praises him for being able to appreciate nature rather than work for capitalism as a whole.

\section {What is the difference between the 'second' and 'first' generation of Romantic poets?}
According to: \footnote{\textcite{epertutti-romanticpoets}}

The first generation poets such as Wordsworth, were in favour of the french revoloution, and their language was easy. They mainly wrote songs and ballads, and used similies and metaphors.
\newline
The second generation poets such as Keats used more classical poems like odes or sonnets, and the language was more difficult. They also saw themsec acalves as prophets.



\newpage
\printbibliography
\end{document}
