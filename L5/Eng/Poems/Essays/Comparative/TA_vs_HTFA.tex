\documentclass[12pt, oneside]{article}
\usepackage[top=1cm, left=2cm, right=1.5cm]{geometry}
\title{Comparative Poems: To Autumn vs Home Thoughts from Abroad}
\author{Jack Maguire}

\begin{document}
\maketitle
\pagenumbering{gobble}

These two poems both describe Nature and it's beauty, however they both do it in very different ways. Important areas to look at are structure and perspective.
\par
\bigskip



But first, the way each Poet describes Nature. In To Autumn (TA), Keats describes the more inanimate side of Nature, choosing to focus on \textit{'all [the] fruit'}, and how it is later harvested in the poem, unlike Browning's focus on wildlife. Keats also focuses lots on the humans in the scene, choosing to describe them as \textit{'sound asleep'}, and innocent. However, in Home Thoughts from Abroad (HTfA), we focus on more of the animate side of nature, the \textit{'Whitethroat build[ing]'}. This helps to demonstrate that whilst TA is very focused on the macro, which is easier to describe asleep, and Home Thoughts From Abroad is focused on the micro. This makes sense, because HTfA is more nostalgic, as shown by the first line where he thinks back to \textit{'be[ing] in England'}. Browning chooses to focus on the \textit{'blossomed pear tree [...] lean[ing] in the field'}. Next, is that Keats also describes the fruits in a wholly different way, as if they were simply play-things of a higher being, \textit{'to swell the gourd'}. We can tell this because of the personification - if the higher power wasn't there to \textit{'plump the hazel shells'}, then they simply would not be right. Compared to HTfA, where when plants are spoken about, they are not personified - \textit{'lowest boughs and the brushwood sheaf'}, and they simply are there. Whilst yes, they do grow, they grow naturally.
\par
\bigskip



Next is the structure, beginning with the pacing. In TA, there are 3 stanzas, each of similar length, with a set rhyme scheme - 'ABAB CDEF GG'. This gives the poem a feeling of regularity, such that this happens every year, which it does. However, in HTfA, there are two stanzas of different sizes, with different line lengths and almost no rhyme scheme (there are rhymes, but where they appear is irregular). This makes it feel like prose, or a stream of conciousness. This also appeals to the reader, as poems that feel too strict can feel alien or strictly regimented, whilst prose resonates with the reader. The rhyme scheme also helps to tell the reader how formal or proper the poem is. In TA, we have a strict rhyme scheme and the poem feels very proper, and if we read it, we can tell that it is - there is lots of Middle English, and speak of higher beings. In HTfA however, if we look at the rhyme scheme, it all looks almost accidental, as if the person was so nostalgic they just started rhyming, and if we read the poem that stays true. Finally for structure, looking at the line lengths, in TA we can see that each line is roughly the same length, and then if we look at the poem, we can feel that each action in each line is given a similar weight, as if the cycle will continue repeating forever, which it will (excluding climate change), as the seasons continue on forever. Then, if we look at HTfA, the line lengths are quite irregular with spots of regularity, implying that this feeling of nostalgia comes and goes, but is irregular with when it does so
\par
\bigskip



Finally, the perspective of the two poems, and how it effects and is effected by the content matter. HTfA is written by and from the perspective of a nostalgic Brit living in Italy, and so we view the whole poem through a lens of nostalgia - very fast, \textit{'And after April, when May follows'}, implying the whole month of April flew by, when in reality it doesn't, and also very slow, talking about the \textit{'Blossoms and dewdrops - at the bent spray's edge'}, which only stay like that for a few days possibly. Then, if we look back to TA, we see that the poem is very regular, as if nothing makes a difference, be it Summer, Autumn or Winter. We see parts of the micro, as if the forces above decide to come visit, but for the most part, we see things faster than they happen. In England, we can also tell we are seeing it through a nostalgic lens because we only ever see the beginnings of things yet to come before we leave, such as the nice \textit{'buttercups'} or the \textit{'Little children's dower'}, whereas in TA, we can see the ends of certain things like the \textit{'last oozings'}, or the \textit{'soft-dying day'}. This helps to show the perspective of the poem, as a God sees both life and death, whereas in Nostalgia, we only remember the positives.

To conclude, both poems present seasons and places in different ways, with both speeding by but differing on what they focus on.


\end{document}
