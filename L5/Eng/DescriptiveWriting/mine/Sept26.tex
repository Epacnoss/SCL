\documentclass[11pt, oneside]{article}
\usepackage[top=1cm, left=2cm, right=1.5cm]{geometry}
\title{Descriptive Writing Essay}
\author{Jack Maguire}

\begin{document}
\maketitle
\pagenumbering{gobble}

\section {Describe a person (real or imaginary) who immediately appears to be either sinister or kind and trustworthy.  Pay particular attention to her/his physical features and mannerisms.}

I was looking around the train station, and everyone kept to themselves as normal. There were a few reading books, some thick and some thin. There were a few in the coffee shop, sipping on drinks that were either too hot or just lukewarm, due to the carelessness of a tired barista or two. There were even a few families, late as it was with tired parents, looking like they walked from the grave, next to small children completely engrossed by their screens. The station was dirty and grimy after a busy bank holiday weekend which would soon end, placing millions back in the toll of monotony.
\par
Then, she arrived. She strode over to a chair in the coffee shop, as if expecting table service, before snapping her thin and dextrous fingers. It was unusually loud, as if practised many times before, and like those many other times, the person she referred to (in this case, an half-asleep barista) came over, trying to shake off the yawns and wake up. "Excuse me miss, -" but before he could finish his sentence, she interrupted and asked for an "ice-cold tea, but with no ice". The barista rushed off, deeply confounded by this woman. She got up, but rather than yawning or putting her hands above her head to stretch, she just stood still, barely breathing and went into a trance-like state.
\par
Eventually, the barista approached, slowly but surely. He was unsure on how to get through to the mysterious woman, and so he tried by tapping her gently, or by saying "miss, miss?". Eventually, he placed down the tea gently on the table in front of her, and she noticed and picked it up. The Barista asked her for £3.50, and she gave him a £50 note, whispering to keep the change. He tried to complain, saying that the £50 note wouldn't be taken anywhere, but she ignored him and sipped her drink.
\par
She then picked up her handbag, had a look inside, and for the first time all night, some might say there was a smile on her face although others might argue that it was a trick of the light. Whatever was in there seemed to satisfy her.
\par
Eventually, a train arrived that suited her - straight back to London Waterloo. I might have sworn that she came here from Euston, which is far closer to Waterloo than Crewe but who knows. Everyone sighed in collective relief.

\end{document}
